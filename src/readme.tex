\documentclass[11pt,a4paper]{article}

\XeTeXlinebreaklocale "zh"
\XeTeXlinebreakskip = 0pt plus 1pt minus 0.1pt

\usepackage[left = 2.8cm, right = 3.2cm, bottom = 3cm]{geometry}
\usepackage[BoldFont,SlantFont]{xeCJK}
\usepackage{graphicx}
\usepackage{verbatim}
\usepackage{fancyvrb}
\usepackage{listings}

\newcommand{\classname}[1]{{\itshape{}#1}}
\setCJKmainfont{FangSong}
%\setCJKfamilyfont{quote}[]{KaiTi}
%\setCJKfamilyfont{normal}[]{FangSong}
%\CJKfamily{normal}

\setcounter{secnumdepth}{0}

\begin{document}
\lstset{
  language=Python,
  showstringspaces=false,
  formfeed=\newpage,
  tabsize=4,
  commentstyle=\itshape,
  basicstyle=\ttfamily,
  morekeywords={models, lambda, forms}
}

\title{Snake AI reference}
\author{Ray}
\maketitle
\section{Introduce}
This is a snake game

A snake can go either up, down, left, right. When it's head touch a food, it eat the food and get the score of the food.
When it's head crashes into any snake's body(including it self) or a block, it die and it's body become food. As a specific case, if two snakes' head
crash together, both die.

\section{To be a participent}
The only things you should care are the \classname{Field} and\classname{Snake} and \classname{FieldObj}.

Let's demonstrate it by examples!

First, suppose field is a instance of Field,
\begin{lstlisting}
	field = Field((20, 10))
\end{lstlisting}
Then you can use the method field.getContentAt(pos) to get the content at pos, it will return a FieldObj.
This class(and it's derived classes) has the members which is useful to you,
\begin{itemize}
	\item FieldObj.type
	\item FieldObj.pos
	\item Food.score
	\item Body.owner
\end{itemize}

Here is more examples,
\begin{lstlisting}
content = field.getContentAt((10, 2))
if content == None:
	# it's an empty place
elif content.type == Field.FOOD:
	print content.score
elif content.type == Field.BODY:
	print content.owner
\end{lstlisting}

All this kind of constants are:
\begin{itemize}
	\item Field.FOOD
	\item Field.BODY
	\item Field.BLOCK
	\item Field.LEFT
	\item Field.DOWN
	\item Field.RIGHT
	\item Field.UP
\end{itemize}

What you actually should do is to derive the base snake class Snake.
Your class should implement the methods:
\begin{description}
	\item[response()] This will be called in each game loop, it should return a direction it choose to go in this loop.
\end{description}

And the these members may be useful to you:
\begin{description}
	\item[body] It's a list, each of it's element is a Body class, a derive class of FieldObj.
	\item[direction] It's one of Field.UP, Field.DOWN, Field.LEFT, Field.RIGHT.
	\item[field] The snake is on this field
	\item[name] The snake's name, shound not conflict with others
\end{description}
See the example code file "stupidAI.py" for more details.
\end{document}
